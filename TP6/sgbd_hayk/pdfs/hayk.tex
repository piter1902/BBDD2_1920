\documentclass{article}
\usepackage[utf8]{inputenc}
\usepackage[margin=0.8in]{geometry}
\usepackage{graphicx}
\usepackage{wrapfig}
\usepackage{eurosym}
\usepackage[hidelinks]{hyperref}




\begin{document}
\section{MySQL}
MySQL es un sistema gestor de bases de datos relaciones de código abierto. 
Es un software gratuito y libre bajo los términos de GNU\footnote{General Public License}. MySQL también se encuentra bajo licencias cerradas como es la de Oracle, quien adquirió a Sun Microsystems, uno de los primero propietarios de MySQL. Tras esta adquisición surgió MariaDB otro RDBMS \footnote{Relational Database Manager System} que surgió debido a la privatización de MySQL.\\

El MySQL es la base de datos de software libre más popular del mundo para distribuir de modo económico aplicaciones web o e\- commerce, procesamiento de transacción online y base de datos integrada confiables.  Es una base de datos integrada, segura para transacción, de conformidad con las propiedades ACID, rollback, recuperación de fallas y traba al nivel de fila. MySQL proporciona facilidad de uso, posibilidad de dimensionamiento y alto desempeño, así como un paquete completo de drivers de base de datos y herramientas visuales para ayudarles a los desarrolladores y DBAs a crear y gestionar sus aplicaciones MySQL. \\

Entre las funcionalidades podemos destacar por ejemplo la capacidad de redimensionamiento para atender a las demandas de cargas de datos y usuarios en crecimiento exponencial. También se dispone de clusters de replicación con auto-recuperación para mejorar la escalabilidad, el desempeño y la disponibilidad.\\
Otra de las funcionalidades destacables es el poder cambiar el esquema online para satisfacer exigencias de negocio que se cambian constantemente. Este gestor también ofrece herramientas de motorización (a nivel aplicación y consumo de recursos) y de autentificación mediante PAM y encriptación.
Otra característica fundamental es el acceso tanto SQL como NoSQL para realizar consultar y operaciones simples y rápidas de Clave-Valor. Finalmente destacar que ofrece soporte en difernetes SO desde Linux, hasta Windows, pasando por FreeBSD y MacOS\\

Cuando hablamos de escalabilidad MySQL ofrece esta solución a través de diferentes métodos. Permite replicación sobre diferentes nodos para distribuir la carga de trabajo. También trata con IDs en las transacciones para facilitar la conmutación por error y la creación de topologías de replicación en anillo. Respecto a los esclavos, son multiprocesos para garantizar la disponibilidad, además que son cash-safe, es decir, permiten recuperarse ante fallos. Finalmente ofrece la herramienta de particionado MySQL Fabric para distribuir los datos en diferentes servidores.\cite{lock}\\

Otro de los pilares fundamentales es la concurrencia. MySQL ofrece un mecanismo de hilos con un buen \textbf{\textit{throughput}}. Provee un hilo por conexión y este hilo no se reasigna hasta que la conexión no acaba, además este método es escalable. Con este mecanismo es suficiente para la mayoria de webs, sin embargo, si se va a tratar con mucho tráfico es recomendable usar la versión enterprise que es ofertada con un \textbf{\textit{Pool de Threads}}. Con esta última funcionalidad el gestor nos provee grupos de thread en función de la demanda, es útil una vez sobrepasamos las 256 conexiones simultaneas a la base de datos.
El control de la concurrencia se realiza en dos niveles, primero a nivel servidor y segundo a nivel almacenamiento. Cuenta con bloqueos de lectura y escritura con gran transparencia. Destacar que tiene bloqueos a nivel de tablas y de filas, esta última ofrece la mayor concurrencia posible, sin embargo, es la menos eficiente. \cite{enterprise}\\


MySQL está disponible en diferentes SO como son: Linux, Windows, Oracle Solaris,  macOS, FreeBSD entre muchas otras
Además ofrece conectores para todos los lenguajes clave de desarrollo que incluyen PHP, Perl, Python, Java, C, C++, C\#, Ruby, Node.js y Go. Cabe destacar que si optamos por la opción de MySQL enterprise podemos acceder a una amplia gama de herramientas de Oracle. \\
En cuanto a comunidad cuenta con un buen apoyo, en StackOverflow podemos ver que existen mas de medio millón de consultas respecto a MySQL.\\
Oracle cuenta con soporte para MySQL, ya que este está más dedicado mas al ámbito de web es fundamental para Oracle para que pueda completar su catálogo. De hecho, el gestor aumentó su popularidad tras la adquisición y el mantenimiento de Oracle.

En cuanto a empresas que usand MySQL destacan los grande como
 \href{https://blog.twitter.com/engineering/en_us/a/2013/new-tweets-per-second-record-and-how.html}{Twitter }, 
\href{https://www.theregister.co.uk/2013/06/27/facebook_tao/}{Facebook},
 \href{http://highscalability.com/blog/2012/3/26/7-years-of-youtube-scalability-lessons-in-30-minutes.html}{YouTube},
  PayPal, Yahoo, Linkedin, Ticketmaster...
Para terminar vamos a hablar del coste. Si optamos por la versión estandar, el coste es de 2000 euros anuales, en cambio, si optamos por la versión enterprise este precio se eleva a 5000 euros anuales la versión de MySQL con hasta 4 sockets en el servidor.\cite{precios}\\

\section{Caché}
Caché es un SGBD comercial que pertenece a InterSystems. Está destinado principalmente a servicios de salud, administración, banca, finanzas, gobierno y otros sectores parecidos. Caché permite usar la base de datos con objetos y con SQL, al igual que permite a los desarrolladores manipular directamente las estructuras de datos almacenados en arrays multidimensionales. Caché fue influenciado por el lenguaje de programación \textit{MUMPS}\footnote{https://en.wikipedia.org/wiki/MUMPS}.\cite{wikiCache}












\newpage
\bibliographystyle{ieeetr}
\bibliography{../references}
\end{document}