\documentclass[11pt,a4paper]{article}
\usepackage[utf8]{inputenc}
\usepackage[spanish]{babel}
\usepackage{amsmath}
\usepackage{amsfonts}
\usepackage{amssymb}
\usepackage[left=2cm,right=2cm,top=2cm,bottom=2cm]{geometry}
\usepackage[hidelinks]{hyperref} 
\usepackage{eurosym}

\begin{document}

\begin{titlepage}
\title{
\begin{Huge}
\textbf{Análisis de Oracle, Cassandra y SQL Server}
\end{Huge}
}
\author{Pedro Tamargo}
\date{\today}
\maketitle
\tableofcontents

\end{titlepage}


\newpage

\section{Oracle}

\emph{Oracle} es un sistema gestor de bases de datos multimodelo (soporta modelo relacional y modelo objeto relacional), perteneciente a la compañía \emph{Oracle Corporation} \cite{WIKI:1} . En 1977 fue fundada la compañía \emph{SDL (Software Development Laboratories)}, este \emph{SGBD} proviene de un proyecto financiado por la \emph{CIA} para diseñar un sistema especial de bases de datos con código clave \emph{``Oracle''} \cite{WIKI:2}. Los fundadores de \emph{SDL} habían leido un artículo en la revista \emph{IBM research} donde se describía una versión preliminar de \emph{SQL} basado en el artículo de \emph{E. F. Codd\footnote{\url{https://es.wikipedia.org/wiki/Edgar_Frank_Codd}}}, donde propone el \emph{modelo relacional}. En 1984, \emph{SDL} adopta el nombre de \emph{Oracle Corporation}.\\

En su versión 19, la instalación de este \emph{SGBD} se puede realizar \emph{on-premise} (en los servidores de la empresa) o en el \emph{cloud} de \emph{Oracle}. Si elegimos la opción \emph{on-premise}, los sistemas operativos en los que es posible instalar este \emph{SGBD} son: \emph{Windows}, \emph{Linux} y \emph{Solaris} \cite{ORA:1}.\\

\emph{Oracle} es una de las bases de datos líderes del mercado en bases de datos operacionales \cite{GART:1}. Cuenta con mucha documentación disponible desde su página web\footnote{\url{https://docs.oracle.com/en/database/oracle/oracle-database/19/lnpls/index.html}} y además cuenta con una comunidad muy activa en la resolución de incidencias y dudas en páginas como \emph{StackOverflow} \cite{STO:1}.\\

El dominio de aplicación de este \emph{SGBD} se encuentra desde el desarrollo de aplicaciones hasta servicios de almacenes de datos \cite{ORA:2}. 
\textbf{{\LARGE Algo sobre el ámbito principal? No sé cual es y nadie dice nada.}}\\

El sistema de licenciamiento de este \emph{SGBD} depende del número de usuarios y el número de procesadores que tenga el servidor donde se ha de instalar. Para cada \emph{``Processor License''} se aplica el \emph{``Core factor''}\footnote{\url{http://www.oracle.com/us/corporate/contracts/processor-core-factor-table-070634.pdf}} y ese es el número de \emph{``Processor Licenses''} necesarios. Para el cálculo de \emph{``Named User Licenses''} se aplicará el máximo entre:

\begin{itemize}
\item $ProcessorLicenses * CoreFactor * NumUserMinimun$
\item $NumUsuarios + NumDispositvos$
\end{itemize}

Donde, $NumUserMinimun$ se corresponde con el número de usuarios mínimos por cada \emph{``Processor License''} \cite{ORA:3}. Los precios de \emph{``Processor License''} y \emph{``Named User Licenses''} en la versión \emph{Enterprise Edition} se corresponden con: 47500\$ y 950\$ \cite{ORA:4}. El resto de funcionalidades no añadidas, como las de almacenes de datos, se adquieren por separado. \\

En el tema de escalabilidad, \emph{Oracle} cuenta con \emph{RAC (Real Application Cluster)}, una tecnología que permite escalar una base de datos a un cluster de servidores ejecutando multiples instancias sobre una misma base de datos de manera transparente al usuario, aprovechando los recursos del clúster. \cite{ORA:5} \\

El control de la concurrencia en este \emph{SGBD} se realiza mediante los mecanismos de bloqueo del mismo para garantizar la consistencia de los datos entre transacciones. \emph{Oracle} se encarga de gestionar los mecanismos de bloqueo automáticamente, de tal forma que no es necesario que el usuario escriba el bloqueo de forma explícita.\cite{ORA:6}
Los mecanismos de bloqueo utilizados por este \emph{SGBD} se dividen en dos grupos, bloqueos exclusivos y bloqueos compartidos. \cite{ORA:7} \\

Para el desarrollo con este \emph{SGBD} se puede utilizar el driver \emph{ODBC} (o \emph{JDBC} si se usa \emph{Java}) desde la propia página de la compañía. \\
El uso de herramientas de soporte complementarias como puede ser \emph{DBeaver\footnote{\url{https://dbeaver.io/}}} y otras opciones de apoyo al desarrollo es posible configurando (si fuera necesario) el driver de conexión a la base de datos. \\
Las copias de seguridad se pueden realizar utilizando la herramienta \emph{RMAN}. Esta herramienta es específica para las bases de datos \emph{Oracle} y permite realizar copias de seguridad sobre las bases de datos y realizar una vuelta a una versión estable si disponemos de una copia de seguridad \cite{ORA:8}. \\

El soporte técnico de este \emph{SGBD} se adquiere por separado, y se relaciona con cada uno de los productos adquiridos. \cite{ORA:4} \textbf{{\LARGE COMPROBAR ESTO}}\\

\newpage 

\section{Apache Cassandra}

\emph{Cassandra} es un sistema gestor de bases de datos \emph{NoSQL (Not Only SQL)}, con almacenamiento basado en columnas. Fue creado desarrollado por \emph{Avinash Lakshman}, uno de los autores de \emph{Amazon's Dynamo\footnote{\emph{Amazon's Dynamo} un conjunto de técnicas que juntas pueden formar un sistema de almacenamiento estructurado de alta disponibilidad o un depósito de datos distribuidos.\cite{WIKI:3}}} por \emph{Facebook} como un proyecto para mejorar la búsqueda en la bandeja de entrada de la plataforma. En 2008, \emph{Facebook} liberó a \emph{Cassandra} como un proyecto open-source en \emph{Google Code}\footnote{\emph{Google Code} fue una plataforma que proveía control de versiones, seguimiento de incidencias para proyectos open-source. Actualmente se encuentra en modo se solo lectura desde Agosto de 2015. \cite{WIKI:4}}. En Marzo de 2009, \emph{Apache} acogió este proyecto, transformándolo en un proyecto de \emph{Apache Incubator} (proyecto en la que los proyectos open-source se pueden convertir en nuevos proyectos de alto nivel de la fundación \emph{Apache} \cite{AP:1}). En Febrero de 2010 se graduó como un proyecto de alto nivel. \cite{WIKI:5}\\

\emph{Cassandra}, al estar programado sobre \emph{Java}, es multiplataforma, es decir, se puede instalar sobre cualquier sistema operativo siempre que cuente con \emph{Java (recomiendan la última versión de Java 8)}. La opción más habitual es utilizar un servidor con SO \emph{Linux}. \cite{AP:2}\\

Este \emph{SGBD} pertenece a \emph{Apache Software Foundation}, una compañía sin ánimo de lucro que desde 1999 ha desarrollado software open-source. \emph{Cassandra} se graduó como un proyecto de \emph{Apache Incubator} en 2010, y desde entonces ha liberado 11 versiones. Sigue manteniendo de forma activa 3 de ellas.\\

El dominio de aplicación de \emph{Cassandra} se corresponde con el de entornos distribuidos donde se puede sacrificar la \emph{consistencia} de los datos frente a la disponibilidad y a la tolerancia a particiones de red (Teorema CAP). Se centra en la disponibilidad y la escalabilidad lineal, es decir, se mejora el rendimiento en relación al número de nodos que se encuentran en la red \cite{WIKI:5}. Esta orientado a entornos donde no existe un nodo maestro, es decir, entornos totalmente distribuidos donde los nodos se comunican por una red \emph{P2P}.\\

Este software se distribuye bajo una licencia \emph{Apache 2}, es una licencia open-source que implica que se puede utilizar par cualquier propósito de forma gratuita. La licencia \emph{Apache 2} es una licencia permisiva ya que la modificación del software distribuido con la misma no tiene por qué mantener esta licencia, exceptuando las partes que no hayan sido modificadas. \cite{WIKI:6}\\

A diferencia de los gestores de bases de datos relacionales, \emph{Cassandra} no sigue las propiedades de transacciones \emph{ACID}, pero en su lugar ofrece transacciones \emph{atómicas}, \emph{aisladas} y transacciones \emph{persistentes} con \emph{consistencia eventual} o modificable, es decir, el usuario puede decidir el nivel de exigencia que quiere fijar sobre la consistencia de una transacción. \cite{DS:1} \\

La herramienta de realización de consultas es \emph{CQLSH}, una herramienta de línea de comandos para interactuar con \emph{Cassandra} utilizando \emph{CQL (Cassandra Query Language)} \cite{AP:3}, un lenguaje para la realización de consultas y manipulación de los datos muy similar a \emph{SQL}. \cite{WIKI:5} \\
Existen drivers (\emph{ODBC} y \emph{JDBC (Java)}) con los cuales podemos realizar el acceso a la base de datos utilizando distintos lenguajes de programación, tales como: \emph{Java}, \emph{Python}, \emph{NodeJS (JavaScript)}, \emph{Dart}, \emph{C++}... \cite{AP:4} \\
Para la administración de los nodos de un clúster, \emph{Cassandra} provee la herramienta \emph{Nodetool}. Esta herramienta brinda la capacidad de realizar tareas de administración sobre los nodos (añadir un nodo a un clúster, modificar la configuración de un nodo...) tanto como las herramientas necesarias para realizar una monitorización sobre el mismo. \cite{WIKI:5} \\

El soporte técnico de este sistema no lo realiza \emph{Apache}, si no que recae sobre empresas de terceros. Por ejemplo, \emph{DataStax\footnote{\url{https://www.datastax.com/}}} ofrece servicios profesionales, servicio técnico y despliegues de este \emph{SGBD}. \cite{APW:1}\\


\newpage


\bibliography{bibliografia}
\bibliographystyle{ieeetr}  % Cambiar por abbrv si da problemas


\end{document}