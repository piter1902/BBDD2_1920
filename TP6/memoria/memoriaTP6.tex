\documentclass{article}
\usepackage[utf8]{inputenc}
\usepackage[margin=0.8in]{geometry}
\usepackage{graphicx}
\usepackage{wrapfig}

\title{Trabajo Práctico 6 - BASES DE DATOS 2}
\author{
  Hayk Kocharyan\\
  757715@unizar.es
  \and
  Pedro Tamargo Allué\\
  758267@unizar.es
  \and
  Jesús Villacampa Sagaste\\
  755739@unizar.es
  \and
  Juan José Tambo Tambo\\
  755742@unizar.es
}

\date{\today}

\usepackage{natbib}
\usepackage{graphicx}
% Para los codigos
\usepackage{listings}

\begin{document}

\maketitle

%\input{insbox}
\tableofcontents

\newpage 

\section{Enunciado}
Para la realización del trabajo se ha elegido el problema del \emph{Gobierno de Aragón}. El problema proporcionado en el enunciado dice lo siguiente:\\

\emph{``El Gobierno de Aragón quiere gestionar información turística de la región para ofrecerla a través de un sitio web donde los usuarios podrán hacer comentarios y recomendaciones.''}
\\

Sobre este enunciado podemos sacar los siguientes requisitos:
\begin{itemize}
\item El sistema deberá almacenar los usuarios.
\item El sistema permitirá almacenar comentarios.
\item El sistema permitirá almacenar recomendaciones.
\item Al ser un sistema online en el que los usuarios podrán publicar sus comentarios y recomendaciones sobre destinos turísticos de Aragón, será importante que el \emph{SGBD} elegido garantice el control de la concurrencia.
\item El presupuesto será el menor posible para que el \emph{SGBD} garantice la calidad del servicio.
\item El sistema deberá ser tolerante a fallos. Priorizaremos la consistencia.
\item El tiempo de respuesta del sistema debe ser mínimo. \textbf{?}
\item El \emph{SGBD} elegido deberá ser un \emph{OLTP(Online Transaction Processing)} ya que recibirá muchas operaciones de lectura y escritura por unidad de tiempo.
\end{itemize}

Dado que \emph{Juan José} es un asco moriremos.\\


\end{document}
