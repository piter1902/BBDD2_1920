\documentclass[11pt,a4paper]{article}
\usepackage[utf8]{inputenc}
\usepackage[spanish]{babel}
\usepackage{amsmath}
\usepackage{amsfonts}
\usepackage{amssymb}
\usepackage[left=2cm,right=2cm,top=2cm,bottom=2cm]{geometry}
\usepackage[hidelinks]{hyperref} 
\usepackage{eurosym}

\begin{document}
	
\begin{titlepage}
\title{
\begin{Huge}
\textbf{Análisis de MongoDB y DB2}
\end{Huge}
}

\author{Juan José Tambo Tambo}
\date{\today}

\end{titlepage}

\newpage

\section{MongoDB}

\emph{MongoDB} es un sistema de base de datos \emph{NoSQL}, orientado a documentos y de código abierto. Los datos se almacenan en estructuras de tipo \emph{BSON}\footnote{Especificación similar a \emph{JSON}} con un esquema dinámico, facilitando y agilizando la integración de los datos en determinadas aplicaciones.\\
Se puede obtener de forma gratuita bajo la licencia de código abierto  \emph{AGPL}\footnote{Licencia pública general de Affero de \emph{GNU}}. También ofrece una licencia comercial que incluye distintas características como integración con \emph{SASL, LADP} o \emph{Kerberos}.\cite{WIKI:1}\\

Entre las principales características del gestor\cite{WIKI:2}, se puede destacar la capacidad de consultas \emph{ad hoc}, con búsqueda por campos, consulta de rangos y expresiones regulares; Su capacidad de indexar cualquier campo en un documento\footnote{Los índices son almacenados en una estructura Árbol-B} y crear índices secundarios, aumentando la eficiencia de las consultas\cite{MDB:1}; Soporta la replicación \emph{primario-copia}, siendo cada grupo de los mismos un \emph{replica set}\footnote{Un \emph{replica set} es un grupo de procesos \emph{mongod} que mantienen en mismo conjunto de datos.\cite{MDB:2}}; La posibilidad de escalar horizontalmente, utilizando \emph{Sharding}\footnote{Método para distribuir datos a través de múltiples máquinas}, permitiendo añadir nuevos nodos al sistema mientras está en funcionamiento.\cite{MDB:3};  Mediante \emph{GridFS}\footnote{Especificación para almacenar y consultar archivos que superan el límite de tamaño de documentos \emph{BSON} de 18 \emph{MB}.\cite{MDB:4}}, \emph{MongoDB} puede ser utilizado como un sistema de archivos con balanceo de carga y tolerante a fallos, aprovechando las características de \emph{MongoDB}. Por último, permite realizar consultas con \emph{JavaScript}\cite{MDB:5}, enviándolas a la base de datos para que sean ejecutadas.\\

Mongo no garantizaba las propiedades \emph{ACID} multidocumento hasta su versión 4.0 (enteriormente se cumplían para un solo documento), aunque posee las siguientes limitaciones\cite{WIKI:3}: 

\begin{itemize}
	\item Bloquea la base a nivel de documento en las escrituras, impidiendo realizar operaciones de escritura concurrentes en el mismo documento. 
	\item El gestor retorna tras escribir en al menos una réplica, lo que no basta para garantizar la durabilidad ni la verificabilidad.\cite{MDB:9}
	\item Si supera los 100 GB de datos, aparecen problemas de rendimiento.\cite{MDB:10}
\end{itemize}

El gestor posee drivers oficiales para diferentes lenguajes de programación: \emph{C, C++, C}\#, \emph{/.NET, Java, JavaScript, Node.js, Perl, PHP, Python, Ruby, Scala, Delphi} y \emph{C++ Builder.}\\

Un apartado importante de MongoDB es \emph{Mongo Shell}\cite{MDB:6}, una interfaz interactiva en \emph{JavaScript} mediante la cual se puede consultar y modificar datos e incluso realizar operaciones de administración. Puede ser utilizado en \emph{Windows, macOS} o \emph{Linux}, lo que convierte a \emph{MongoDB} en un gestor multiplataforma.\\

\emph{MongoDB} proporciona una serie de productos en la nube para poder almacenar y gestionar instancias del mismo, como lo son \emph{MongoDB Stitch, MongoDB Atlas Data Lake} o \emph{MongoDB Cloud Manager}, entre otros.\cite{MDB:7}

Además, el gestor posee una gran serie guías para cada una de sus funcionalidades en su página web\cite{MDB:8}, así como un soporte técnico al que se contactar mediante correo o vía telefónica\footnote{https://www.mongodb.com/contact}.

\newpage

\section{DB2}

\emph{IBM DB2} es un gestor de bases de datos relacional orientada a empresas capaz de gestionar datos estructurados y no estructurados en entornos públicos, privados y locales.\cite{DB2:1} Actualmente es una marca comercial, propiedad de \emph{IBM}, aunque posee una versión gratuita, \emph{DB2 Express-C}\cite{DB2:2}, que permite desarrollar aplicaciones con las funcionalidades básicas de \emph{DB2}.\\

Entre sus principales características destacan la capacidad de manejar objetos de hasta 2 GB, definición de funciones y datos/tipos por parte del usuario, mecanismos para garantizar la integridad referencial, \emph{SQL} recursivo\footnote{https://geeks.ms/jirigoyen/2009/05/22/recursividad-con-sql-server/}, tratamiento de archivos multimedia, procesamiento paralelo, commit en dos fases y \emph{backup} on-line y offline.\\
Además permite la conexión directa de una aplicación escrita en \emph{Java}con la base de datos utilizando JDBC,  pudiendo ser una base local o remota.\cite{DB2:3}\\

El gestor utiliza \emph{prueXML}\footnote{Función de almacenamiento XML nativo.} para almacenar documentos en formato \emph{XML}, para realizar operaciones y búsquedas dentro del mismo e integrarlo con búsquedas relacionales.\cite{WIKI:4} De esta manera se puede realizar consultas de tipos de datos no tradicionales. Además utiliza un sistema multiprocesador \emph{SMP} simétrico y un sistema de procesador paralelo masivo, lo que permite su alto rendimiento. \\

Se trata de un sistema altamente escalable al dar soporte a volúmenes de datos de \emph{petabytes} y acelerar las consultas, pudiendo escalar desde un \emph{laptop} a \emph{clusters} de servidores empresariales paralelos.\cite{DB2:4}.
Asimismo, \emph{DB2} ofrece múltiples opciones de despliegue (híbrido, \emph{on premise}, cloud), lo que da soporte a un amplio rango de cargas de trabajo.

\emph{DB2} utiliza un sistema de \emph{commit-rollback} diferente a otros gestores. Todos cambios en la base se almacenan en el \emph{Log Buffer}, el cual escribe en disco cuando se realiza un \emph{commit}. Un \emph{commit} no es correcto hasta que se ha copiado toda información en disco.\cite{DB2:5}\\

Las plataformas que admite el gestor son \emph{AIX, HP-UX, UNIX, Linux, Solaris y Windows}, además de poseer una versión para \emph{Z/OS}\footnote{Sistema actual de las computadoras de \emph{IBM}}, la cual está optimizada para \emph{SOA, CRM} y almacenamiento de datos.\cite{DB2:6}\\

Por último, el gestor ofrece una serie de herramientas de administración, gestión de rendimiento, recuperación, réplicas y aplicaciones. La mayoría de estas herramientas proporcionan una interfaz gráfica de usuario (\emph{GUI}) y una interfaz \emph{ISPF}\footnote{Interactive System Productivity Facility} que permiten al usuario realizar la mayoría de tareas de \emph{DB2} de forma interactiva.\cite{DB2:7}

\newpage

\addcontentsline{toc}{section}{Referencias}
\bibliography{bibliografia}
\bibliographystyle{ieeetr}

\end{document}