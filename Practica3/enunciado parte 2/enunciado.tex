\documentclass[11pt,a4paper]{article}
\usepackage[utf8]{inputenc}
\usepackage[spanish]{babel}
\usepackage{amsmath}
\usepackage{amsfonts}
\usepackage{amssymb}
\usepackage[left=2cm,right=2cm,top=2cm,bottom=2cm]{geometry}

\begin{document}

\title{{\Huge \textbf{Enunciado parte 2 - Bases de Datos 2}}}
\author{Pedro Tamargo Allué}
\date{\today}
\maketitle

\section{Enunciado}

Se desea diseñar una base de datos para gestionar la información inherente a un hospital. En el hospital trabajan médicos y enfermeros. Los médicos tienen asignados una serie de pacientes. Los enfermeros tienen asignada una planta.\\
%Los pacientes ingresan en el hospital a partir del servicio de urgencias, y tras examinarles se les asigna a una planta.\\
La información sobre los trabajadores del hospital deberá contener información sobre su DNI, nombre, apellidos, fecha de nacimiento, dirección, teléfono de contacto. Los médicos deberán almacenar su especialidad y su número de colegiado. Los enfermeros almacenarán el servicio al que están asignados (planta, hematología, diálisis...).\\
Nuestro hospital está dividido en plantas, cada planta alberga pacientes ingresados y almacena la información relativa al uso destinado para esa planta.\\
Los pacientes almacenan su DNI, nombre, apellidos, fecha de nacimiento, dirección y teléfono. Disponemos de un historial clínico que almacena la información sobre los diagnósticos realizados a los pacientes.

\end{document}