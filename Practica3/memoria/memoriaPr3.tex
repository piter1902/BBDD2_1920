\documentclass{article}
\usepackage[utf8]{inputenc}
\usepackage[margin=0.8in]{geometry}
\usepackage{graphicx}
\usepackage{wrapfig}

\usepackage{natbib}
\usepackage{graphicx}
% Para los codigos
\usepackage{listings}
\usepackage[spanish]{babel}
\usepackage[hidelinks]{hyperref}


\begin{document}

\begin{titlepage}
	\title{
		\begin{Huge}
			Practica 3 - BASES DE DATOS 2
		\end{Huge}
	}
	\author{
	  Hayk Kocharyan\\
	  757715@unizar.es
	  \and
	  Pedro Tamargo Allué\\
	  758267@unizar.es
	  \and
	  Jesús Villacampa Sagaste\\
	  755739@unizar.es
	  \and
	  Juan José Tambo Tambo\\
	  755742@unizar.es
	}
	\date{\today}
	
	\clearpage\maketitle
	\thispagestyle{empty}
	\tableofcontents
	
\end{titlepage}

\newpage 

\section{Esfuerzos invertidos}
Los esfuerzos invertidos por cada integrante del equipo son:
\begin{itemize}
\item Hayk:

\item Juanjo:	

\item Jesús:

\item Pedro: 

\end{itemize}

\section{Parte 1 - Esquema conceptual y lógico de la base de datos relacional diseñada en la práctica anterior}

\textbf{Mostrar aqui el ER de la práctica anterior y contar su traducción a SQL diciendo las decisiones de diseño de forma DETALLADA.}

\section{Parte 1 - Determinación del esquema lógico y conceptual de la base de datos Oracle a integrar}

\textbf{Esquema ER de la base de datos a integrar y PONER LAS CONSULTAS CON LAS QUE HEMOS SACADO LAS COSAS.}\\
\textbf{Igual estaría bien comentar de donde hemos sacado la información.}

\section{Parte 1 - Mejoras sugeridas para la base de datos a integrar}

\textbf{Hablar aqui de los problemas que hemos encontrado y lo que supone que esté mal.}

\section{Parte 1 - Definición e implementación del esquema global en Oracle}

\textbf{Poner aqui el esquema ER y decir las decisiones de diseño (y el por qué).}

\section{Parte 2 - Enunciado de un problema de diseño de bases de datos}

\textbf{Poner aqui ese enunciado que hemos redactado}

\section{Parte 2- Esquema conceptual para el problema enunciado}

\textbf{Poner aquí los dos esquemas ER y explicar que se ha tomado esta decisión para dificultar la integración.}

\section{Parte 2 - Esquema lógico 1 para el problema enunciado}

\textbf{Pues explicar las decisiones de diseño de este (CHUS Y RATAMBO).}

\section{Parte 2 - Esquema lógico 2 para el problema enunciado}

\textbf{Pues explicar las decisiones de diseño de este (PEDRO Y HAYK).}

\section{Parte 2 - Esquema global y su implementación en PostgreSQL}

\textbf{Puse aquí tenemos la incertidumbre de si querrán que sean vistas o una base de datos completa, pero de todas formas habrá que explicar las decisiones de diseño.}

\section{Actualizaciones de datos sobre el esquema global}

\textbf{Viabilidad? Pues depende de donde quieras guardarlo. Pero esto habrá que hablar.}


\end{document}
